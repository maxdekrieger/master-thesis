% !TEX root = document.tex

\chapter{\label{chap:evaluation}Evaluation}

In this chapter we will evaluate the results of the work done in this thesis. First we will assess the correctness of the modernized WebDSL front-end by defining what correctness means in absence of a formal specification and evaluating accordingly. Next, we will share, inspect and reason about the performance of the new parser and static analyses. Lastly, we conclude this chapter by discussing the usability of the modernized implementation in practice.

\section{\label{sec:correctness}Correctness}

  \begin{itemize}
    \item Defining correctness in absence of a formal specification
    \item How correct is the implementation WebDSL
    \item Explain correctness
    \item Edge cases
  \end{itemize}

\section{\label{sec:performance}Performance}

  \begin{itemize}
    \item Explain metrics and methods
    \item Results
    \item Discuss results
  \end{itemize}

\section{\label{sec:usability}Usability}

  \begin{itemize}
    \item Lack of user-friendliness of the error messages generated by Statix
    \item Can the WebDSL Statix specification be used as formal specification?
    \item Maintainability of the Statix and SDF3 codebase
  \end{itemize}
