% !TEX root = document.tex

\chapter{\label{chap:related-work}Related work}

  \section{Statix Case Studies}

    \subsection{A Constraint Language for Static Semantic Analysis Based on Scope Graphs}

      \subsubsection{LMR (Language with Modules and Records)}

        \begin{itemize}
          \item Module system
          \item Definitions (nominal records, functions, variables, etc)
          \item Expressions
          \item Constants
        \end{itemize}

    \subsection{Scopes as Types}

      \subsubsection{Simply-typed Lambda Calculus with Records}

        \begin{itemize}
          \item Records
          \item Structural subtyping
        \end{itemize}

      \subsubsection{Featherweight Java}

        \begin{itemize}
          \item Classes
          \item Nominal subtyping
        \end{itemize}

      \subsubsection{System F}

        \begin{itemize}
          \item Parametric types
          \begin{itemize}
            \item Solved by lazily duplicating the parametric type structure when an instantiation is made with a concrete type
          \end{itemize}
        \end{itemize}

    \subsection{Knowing when to ask: sound scheduling of name resolution in type checkers derived from declarative specifications}

      \subsubsection{Java (subset)}

        \begin{itemize}
          \item Packages and imports
          \begin{itemize}
            \item Solved remote extension by using a mixin-pattern: compilation units query for all other compilation units within the same package and make their types accessible by adding import edges.
          \end{itemize}
        \end{itemize}

      \subsubsection{Scala (subset)}

        \begin{itemize}
          \item Accessibilty of definitions: local definitions are available in the surrounding scope, whereas imported definitions are available in the subsequent scopes.
          \begin{itemize}
            \item This also implements forward referencing
            \item Possible inspiration for WebDSL action name resolution
          \end{itemize}
        \end{itemize}

      \subsection{LMR/Rust}

        \begin{itemize}
          \item Imports that do influence their own resolution (unordered imports).
        \end{itemize}

  \section{Papers to investigate}

    \begin{itemize}
      \item Type errors for the IDE with Xtext and Xsemantics
      \begin{itemize}
        \item https://www.degruyter.com/document/doi/10.1515/comp-2019-0003/html
        \item 2019, journal Open Computer Science
        \item Implements typechers for two small languages with the Xtext language workbench
        \item Describes what to pay attention to when implementing a typechecker (error recovery, useful error messages, etc.)
        \item Key difference: This thesis contains a larger case study and is written in Spoofax meta-languages
      \end{itemize}
      \item Xbase: implementing domain-specific languages for Java
      \begin{itemize}
        \item https://dl.acm.org/doi/abs/10.1145/2480361.2371419
        \item 2012, GPCE '12 (Generative Programming and Component Engineering)
      \end{itemize}
      \item Migrating custom DSL implementations to a language workbench (tool demo)
      \begin{itemize}
        \item https://dl.acm.org/doi/abs/10.1145/3276604.3276608
        \item 2018, SLE '18 (Software Language Engineering)
      \end{itemize}
      \item The State of the Art in Language Workbenches
      \begin{itemize}
        \item https://link.springer.com/chapter/10.1007/978-3-319-02654-1\_11
        \item 2013, SLE '13 (Software Language Engineering)
        \item 
      \end{itemize}
      \item Evaluating and comparing language workbenches: Existing results and benchmarks for the future
      \begin{itemize}
        \item https://www.sciencedirect.com/science/article/pii/S1477842415000573
        \item 2015, journal Computer Languages Systems \& Structures
      \end{itemize}
      \item Towards a Spreadsheet-Based Language Workbench
      \begin{itemize}
        \item https://ieeexplore.ieee.org/abstract/document/9643797
        \item 2021, MODELS-C '21 (Model Driven Engineering Languages and Systems Companion)
      \end{itemize}
      \item Language Workbench Support for Block-Based DSLs
      \begin{itemize}
        \item https://core.ac.uk/download/pdf/301639649.pdf
        \item 2018, BLOCKS+ in SPLASH '18 (Systems, Programming, Languages and Applications: Software for Humanity)
      \end{itemize}
    \end{itemize}