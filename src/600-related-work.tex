% !TEX root = document.tex

\chapter{\label{chap:related-work}Related work}

  \section{Papers}

    \begin{itemize}
      \item Static consistency checking of web applications with WebDSL (Hemel, Groenewegen, Kats, Visser). JSC 2011.
      \begin{itemize}
        \item Importance of early feedback to developer.
        \item Different inconsistencies that can be checked and how they are often checked.
        \item Importance of linguistic integration to enable consistency checking at compile time.
        \item How WebDSL is designed to enable early reporting of inconsistencies.
      \end{itemize}

      \item Links: Web Programming Without Tiers (Cooper, Lindley, Wadler, Yallop). FCMO 2006.
      \begin{itemize}
        \item Introduces the Links programming language.
        \item Links generates all tiers of web application: client-side HTML and JavaScript, server-side OCaml and SQL for the database.
        \item Links tackles the impedance mismatch problem of different input/output provided and expected by the different tiers.
        \item Links is a strict, typed, functional language with all state saved in the client-side.
      \end{itemize}

      \item Ur/Web: A Simple Model for Programming the Web (Chlipala). POPL 2015.
      \begin{itemize}
        \item Introduces the Ur/Web programming language.
        \item Similar to Links, Ur/Web is a functional programming language that generates code for all tiers of a web application.
        \item Ur/Web introduces encapsulation and simple concurrency.
        \item Encapsulation: treating key pieces of web applications as private state (?).
      \end{itemize}
    \end{itemize}

  \section{Papers to investigate}

    \begin{itemize}
      \item Evolution of the WebDSL runtime
      \begin{itemize}
        \item About the evolution of a programming language
        \item Contains WebDSL details
        \item Key difference: This thesis focusses on front-end of the WebDSL language, not the back-end
      \end{itemize}

      \item Scopes as Types
      \begin{itemize}
        \item About declaratively specifying static semantics
        \item Contains Statix details and case studies
        \item Key difference: This thesis contains a larger case study with more language constructs and different (practically motivated) requirements
      \end{itemize}

    \end{itemize}

    Maybe?:
    \begin{itemize}
      \item Type errors for the IDE with Xtext and Xsemantics
      \begin{itemize}
        \item https://www.degruyter.com/document/doi/10.1515/comp-2019-0003/html
        \item Implements typechers for two small languages with the Xtext language workbench
        \item Describes what to pay attention to when implementing a typechecker (error recovery, useful error messages, etc.)
        \item Key difference: This thesis contains a larger case study and is written in Spoofax meta-languages
      \end{itemize}
    \end{itemize}