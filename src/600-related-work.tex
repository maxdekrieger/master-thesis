% !TEX root = document.tex

\chapter{\label{chap:related-work}Related work}

  In this chapter we discuss work related to compiler front-ends of domain-specific languages. First, we examine published research related to WebDSL that address its grammar and static analysis. Second, we examine work concerning SDF3 and Statix and recent case studies using these meta languages. To conclude, we discuss related work on alternative approaches for implementing the front-end of domain-specific languages.

  \section{WebDSL}

    The paper of Visser (\citeyear{Visser2007}) introduced WebDSL as a case study of Stratego. The work lists most language constructs of the first WebDSL version, combined with SDF2 snippets of their syntax and Stratego code that demonstrates how code is generated for the WebDSL construct. Despite type checking not being mentioned for all language constructs, an extensive example plus implementation in Stratego is given for typechecking a for-loop and the declaration and resolving of a variable is WebDSL. Other work by Groenewegen and Visser present linguistically integrated extensions of WebDSL regarding access control (\citeyear{GroenewegenV08}) and data validation (\citeyear{GroenewegenV13}).

    The work of Hemel et al. (\citeyear{Hemel2011}) identifies static consistency checking as lacking for web frameworks that were modern at the time and they provide an analysis of how the web frameworks deal with certain consistency checks. Hemel et al. argue that domain-specific languages should be designed for consistency checking, providing a deep-dive on WebDSL as example. Additionally, they present how to perform such consistency checks with Stratego, essentially explaining an early version of the current WebDSL static analysis in Stratego.

    The latest publication on WebDSL by Groenewegen et al. (\citeyear{Groenewegen2020}) reflects on the WebDSL language as a whole, and provides an experience report of using WebDSL for over 10 years for increasingly ambitious applications.

  \section{Statix}

    Scope graphs were introduced by Neron et al. (\citeyear{Neron2015}) as a language-independent framework for describing name binding in programming languages. Van Antwerpen et al. (\citeyear{AntwerpenNTVW16}) build upon this work and extend the scope graph framework with generalized edge labels and introduce a constraint language with a solver that is able to express name binding and typing constraints. In a subsequent publication, Van Antwerpen et al. (\citeyear{VanAntwerpen2018}) further extend the scope graph framework to increase the range of language constructs that can be modelled, such as parameterized types. Additionally, this work introduces Statix as a declarative language to specify type systems. In later work, the performance of Statix was boosted by Van Antwerpen and Visser (\citeyear{AntwerpenV21}) through the introduction of the concurrent Statix solver, and by Zwaan et al. (\citeyear{ZwaanAV22}) who introduced an incremental Statix solver.

    \subsection{Case Studies using Statix}

      The paper that introduced scope graphs by Neron et al. (\citeyear{Neron2015}) contains several illustrations on how scope graphs can model the name binding structure of programs. In the paper, the authors illustrate how to model various concepts of the Language with Modules and Records (LMR) such as let-bindings and, unsurprisingly, modules and records. The extended version (\citeyear{TUD-SERG-2015-001}) contains examples on definition-before-use, Java packages and imports and C\# partial classes. Van Antwerpen et al. (\citeyear{AntwerpenNTVW16}) use the LMR example again, but in addition to showing the scope graph, they list the typing constraints for certain snippets, and show an algorithm with rules that dictate how to traverse any LMR program and generate constraints.

      The work that introduced Statix \autocite{VanAntwerpen2018} contained case studies on simply-typed lambda calculus that shows records and structural subtyping, Featherweight Java that presents classes and nominal subtyping. Lastly, the paper contains a case study on System F that illustrates Statix' ability to deal with parametric polymorphism.

      The research of Rouvoet et al. (\citeyear{RouvoetAPKV20}) on sound scheduling of name resolution queries also contains various case studies of languages used in practice, modelled in MiniStatix: the core constraint language of Statix, with some extras implemented in Haskell. The case studies are on a subset of name resolution for Java and scala, and the whole of LMR. In their evaluation they used a combination of valid and invalid programs, similar to this thesis. In contrast to this thesis, Rouvoet et al. do not use error message expectations, simply a fail or succeed expectation but the aim of their evaluation was different than ours. 

      Van Antwerpen et al. (\citeyear{AntwerpenV21}) implement a subset of Java in Statix to evaluate their work on real-world codebases, similar to what we show in this thesis. However, their work is not evaluated on erroneous programs, as their goal is to benchmark the parallel Statix solver in terms of run time.

      In addition to published research, there are multiple Master's theses that contain Statix case studies:

      \paragraph{Aerts (\citeyear{Aerts2019})} In his Master's thesis, Aerts described, implemented and evaluated an approach for incrementalizing Statix based on separate compilation and dynamic dependency detection. In the evaluation, Aerts implemented a simplified version of Java in Statix that focusses on type-dependent name resolution.

      \paragraph{Zwaan (\citeyear{Zwaan2021})} Zwaan provided a runtime that is able to handle composed Statix specifications. He validated his work by integrating Statix specifications of a small subset of SDF3 (Mini-SDF) and Stratego (Mini-STR). Additionally, his Master's thesis contains another case study where the specifications of a small toy-language with expressions, record types, functions and modules (Mod) and a subset of SQL (Mini-SQL).

      \paragraph{Wilms (\citeyear{Wilms2022})} In Wilms' Master's thesis, he introduced PIE DSL 2, a successor of domain specific language accompanying incremental build system PIE \autocite{KonatSEV19}. As part of the PIE DSL 2 implementation, Wilms implemented the static semantics in Statix which modelled many interesting features such as a module system similar to Java, class inheritance and parameterized types.

  \section{Alternative Approaches}

    The Spoofax language workbench provides meta-DSLs such as SDF3, Statix and Stratego for implementing all aspects of a domain-specific language. However, many alternative approaches exist.

    One such approach to defining a static analysis is by using Datalog. Datalog is a subset of Prolog and similar to Statix, it is a declarative logic programming language. Where Statix builds and queries a scope graph, Datalog builds and queries a deductive database. Recent work by Pacak et al. (\citeyear{PacakES20}) utlize the performance of Datalog's incremental solvers by expressing algorithmic typing rules in Datalog.

    Xtext \autocite{EfftingeVoelter2006} is a language workbench that generates with a heavy focus on generated tooling, such as a compiler infrastructure and IDE support, based on the Xtext grammar language. Xtext utilizes the Eclipse Modeling Framework \autocite{Steinberg2009EMF} and provides an API to customize behaviour of the generated tooling using dependency injection \autocite{eysholdt2010xtext}. For example, this customization allows for changing the scoping and name binding rules, or adjusting the code generator.

    Rascal is a metaprogramming language, developed by the CWI SWAT group (\citeyear{KlintSV09}). A commonality of Spoofax and Rascal is that both are in a sense based on the ASF+SDF formalism \autocite{Klint93}. A complete language implementation can be defined using Rascal, including a generated parser, static analysis and code generation. Rascal supports the generation of constraints for program analysis, but does not provide a built-in constrait solver, as they believe hand-crafted specialisations are important for making program analysis scale. In the paper titled \textit{Rascal, 10 years later} (\citeyear{KlintSV19}), Klint et al. mentions constraint-based type checking as an ongoing project.

    The MPS platform by Jetbrains \autocite{Dmitriev2004LanguageOP} is a commercial language workbench with a focus on projectional editing, using representations such as tables and graphs. The first language developed in JetBrains MPS was BaseLanguage, a dialect of Java. The other language definition DSLs of MPS build on BaseLanguage \autocite{PechSV13}. Similar to Xtext, MPS provides many IDE functionality such as type checking and even code completion out-of-the-box. A recent experimental feature of JetBrains MPS named CodeRules\footnote{https://jetbrains.github.io/mps-coderules/about} allows for type inference and type checking using logical programming with constraints, similar to Statix.
