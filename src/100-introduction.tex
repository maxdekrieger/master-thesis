% !TEX root = document.tex

\chapter{\label{chap:introduction}Introduction}

  \subsubsection{Broad Picture}
  Many different programming languages exist, with many different properties and advantages. (\emph{TO-DO: this is crap, need something better})

  \subsubsection{Programming Langauge Front-end Introduction}
  The implementation of a programmig languages can be seperated into two parts: the front-end and the back-end. The front-end is the part of the programming language with which the user interacts (the syntax, early feedback using analysis results) and the back-end is the part that makes the programming language operational (optimization, code generation). While the back-end of a programming langauge makes it work, the front-end defines how a user experiences the progrmaming language (\emph{TO-DO: read papers about this}). 

  \subsubsection{WebDSL}
  Programming languages are constantly evolving, requiring updates to its specification and implementation. One such language is WebDSL. WebDSL is a domain-specific language for developing web applications, developed at the Delft University of Technology.

  \subsubsection{Problem Description}
  Because of its academic nature, many research projects added features to the language, all contributing to the success of existing WebDSL applications. The downside of having many different contributors adding new features, is that the development experience that comes from the front-end leaves much to be desired. (\emph{TO-DO: too harsh?}) Currently, the WebDSL implementation is composed of multiple definitions in meta-DSLs supported by the Spoofax language workbench: the syntax is defined in SDF2 and the desugaring, typechecking, optimization and code generation is defined in the term transformation language Stratego. The interleaving of all the latter processes in the same Stratego environment poses a threat to the readability and maintainability of the WebDSL language.
  
  % Contributions
  \section{Contributions}
    In this thesis, we will be focusing on modernizing the WebDSL front-end, by implementing the syntax definition in SDF3 and the static analyses in Statix and documenting the challenges posed by this process. In this work, the following contributions are made:

    \begin{itemize}
      \item We present an implementation of the WebDSL grammar in SDF3.
      \item We present an implementation of the WebDSL static analyses in Statix.
      \item We assess the completeness and performance of the WebDSL SDF3 and Statix implementation.
      \item We provide qualitative feedback about the development process with SDF3 and Statix.
    \end{itemize}

  % Outline
  \section{Outline}
    The rest of this thesis is structured as follows. In chapter 2 we describe WebDSL, its features and its current implementation. Next, chapter 3 and 4 go in detail about the new implementation of the WebDSL front-end in SDF3 and Statix respectively. The result of this implementation is evaluated in chapter 5 and compared with related work in chapter 6. Finally, chapter 7 concludes this thesis.
