% !TEX root = document.tex

\chapter{\label{chap:introduction}Introduction}

  % Broad picture
  Many different programming languages exist, with many different properties and advantages.

  % Programming langauge front-end introduction
  The implementation of a programmig languages can be seperated into two parts: the front-end and the back-end. 

  The front-end is the part of the programming language with which the user interacts (the syntax, early feedback using analysis results) and the back-end is the part that makes the programming language operational (optimization, code generation).

  While the back-end of a programming langauge makes it work, the front-end defines how a user experiences the progrmaming language. % TO-DO: need citation here

  % WebDSL introduction
  Programming languages are constantly evolving, requiring updates to its specification and implementation. One such language is WebDSL.

  WebDSL is a domain-specific language for developing web applications, developed at the Delft University of Technology.

  % Problem description
  Because of its academic nature, many research projects added features to the language, all contributing to the success of existing WebDSL applications.

  The downside of having many different contributors adding new features, is that the development experience that comes from the front-end leaves much to be desired. % too harsh?

  Currently, the WebDSL implementation is composed of multiple definitions in meta-DSLs supported by the Spoofax language workbench: the syntax is defined in SDF2 and the desugaring, typechecking, optimization and code generation is defined in the term transformation language Stratego.

  The interleaving of all the latter processes in the same Stratego environment poses a threat to the readability and maintainability of the WebDSL language.

  % Goal
  In this thesis, we will be focusing on modernizing the WebDSL front-end, by implementing the syntax definition in SDF3 and the static analyses in Statix.

  % Contributions
  \section{Contributions}
    This thesis provides the following contributions:

    \begin{itemize}
      \item We present the biggest case study for SDF3 and Statix thusfar.
      \item We present a new, modernized WebDSL front-end.
    \end{itemize}

  % Outline
  \section{Outline}
    The rest of this thesis is structured as follows. In chapter 2 we describe WebDSL and its features. Next, in chapter 3 we will give an overview of what a modernized front-end is, and what our solution looks like. Chapter 4 and 5 go in detail about the implementation of the WebDSL front-end in SDF3 and Statix respectively. The result of this implementation is evaluated in chapter 6 and compared with related work in chapter 7. Finally, chapter 8 concludes this thesis.
