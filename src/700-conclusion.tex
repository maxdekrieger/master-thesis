% !TEX root = document.tex

\chapter{\label{chap:conclusion}Conclusion}

  In this thesis, we have presented a new front-end for WebDSL. WebDSL is a domain-specific language for web programming, inspired by multiple programming language paradigms. WebDSL is used to create applications such as WebLab and conf.researchr.org, which have thousands of daily users.

  We have shown the conversion of the WebDSL grammar from an SDF2 specification to an SDF3 that is disambiguated without post-parse filters, and where the definition of sorts and constructors can be reused for Statix. The grammar formalism SDF3 generates a parse table which can be executed to efficiently transform textual programs into abstract syntax trees that are used in subsequent components of the compilation chain, such as static analysis.

  Next, we presented the static semantics of WebDSL modelled in Statix. Statix is a declarative constraint-based programming language using the concept of scope graphs to model program structures and types. Statix comes with a built-in constraint solver that schedules the constraints in a sounds way, builds and queries the scope graph and is able to show error messages for failing constraints.

  The challenges of implementing a modernized WebDSL front-end are documented in this thesis, and we provided qualitative feedback on how to further improve the meta-DSLs SDF3 and Statix. The most notable challenges include but are not limited to:
  
  \begin{itemize}
    \item Disambiguating the WebDSL grammar without the use of post-parse filters \texttt{\{prefer\}} and \texttt{\{avoid\}};
    \item Modifying the grammar such that it adheres to the requirements of the Statix signature generator;
    \item Expressing language constructs such as type extension, overloading, generated definitions and static code template expansion in Statix;
    \item Expressing the WebDSL module system in Statix and defining a revised module system;
    \item Reducing the run time of the WebDSL static analysis in Statix from over 8 hours to 4 seconds on real world applications.
  \end{itemize}

  Lastly, the resulting modernized front-end of WebDSL was evaluated in terms of correctness and run time performance using large test suites and WebDSL applications that are used in practice.

  \section{\label{sec:answering-research-questions}Research Questions}

    The aim of this thesis was to answer the following research questions.

    \begin{itemize}
      \item \textbf{RQ1} Is it possible to define the WebDSL syntax in SDF3, without the use of post-parse filters?
      \item \textbf{RQ2} How does the run time efficiency of the parser generated from WebDSL in SDF3 compare to the current WebDSL parser generated from SDF2?
      \item \textbf{RQ3} How does the maintainability of WebDSL in SDF3 compare to the WebDSL syntax definition in SDF2?
      \item \textbf{RQ4} Is a Statix implementation of the WebDSL static semantics able to catch the same errors and warnings as the current implementation in Stratego?
      \item \textbf{RQ5} How does the run time efficiency of the WebDSL static semantics in Statix compare to the current WebDSL implementation?
      \item \textbf{RQ6} How does the maintainability of the WebDSL static semantics in Statix compare to the current implementation?
    \end{itemize}

    \paragraph{RQ1} In \cref{chap:sdf3} we showed that it is possible to define the WebDSL syntax in SDF3 without the \texttt{\{prefer\}} and \texttt{\{avoid\}} annotations. Doing so required the addition of sorts for disambiguation purposes, which is not idiomatic. Additionally, indexed non-transitive priority rules were necessary to get rid of all observed ambiguities. In \cref{subsec:eval-sdf3-correctness} we showed that the WebDSL syntax test suite, which contains regression tests from production bugs can be parsed by the parser generated from the WebDSL SDF3 definition.

    \paragraph{RQ2} In \cref{subsec:eval-sdf3-performance} we showed that the parse table generated from the WebDSL SDF3 definition is larger than the parse table generated from the current WebDSL syntax definition in SDF2. Even though the described grammar did not change, it is implemented differently using more sorts and constructors, leading to an increase in parse table size which resulted in a slightly slower parser.

    \paragraph{RQ3} While the maintainability of the WebDSL SDF3 definition is subjective, we argue that it is more readable due to the increased similarity to other grammar formalisms in SDF3, such as EBNF. Additionally, the implementation without post-parse filters does increase the directness of finding an ambiguity in the WebDSL SDF3 definition. However, the compliance with Statix Signature Generator and the disambiguation without post-parse filters introduce additional sorts which unnecessarily complicate the grammar. For an objective answer to this question, user studies are required.

    \paragraph{RQ4} The Statix definition the WebDSL static semantics is not able to report the same errors and warnings as the current implementation in Stratego. For example, Statix is missing String manipulation which is necessary to encode certain WebDSL features such as static code template expansion. In \cref{subsec:eval-statix-correctness} we use the analysis test suite of the current implementation, which contains concrete test expectations, namely if an error should be shown, what the error should be and how many errors should be shown. From this test suite, 302 out of 512 tests succeed and a discussion of the failed tests can be found in \cref{subsubsec:eval-statix-failed-tests}. When the test suite is simplified to test whether correct programs pass and incorrect programs show an error, 416 out of the 512 tests succeed.

    \paragraph{RQ5} \Cref{subsec:eval-statix-performance} that the static analysis generated from the WebDSL Statix specification is slightly slower than the current implementation in Stratego. While earlier iterations took over 8 hours to complete, \cref{subsec:revised-module-system} shows how we reduced its run time to several seconds.
    
    \paragraph{RQ6} The meta-language Statix is developed specifically for implementing the static semantics of programming languages, in contrast to Stratego which is broadly applicable. The Statix syntax resembles that of formal inference rules and makes the individual rules more readable than their Stratego implementation. However, encoding the WebDSL static semantics in Statix requires many boilerplate rules and code duplication due to a lack of a standard library or string manipulation. For an objective answer to this question, user studies are required.

  \section{\label{sec:future-work}Future work}

    While the modernized WebDSL front-end using SDF3 and Statix is promising, there are many possibilities for improving and extending the work shown in this thesis.

    \paragraph{Increased Engineering Effort} We argue that the correctness of the modernized WebDSL static analysis scales with the engineering effort put in to the Statix specification (see \cref{subsec:eval-statix-correctness}). For the development of web applications with WebDSL, catching more erroneous programs before compilation and reporting telling error messages is essential, especially since the current implementation in Stratego is capable of doing so.

    \paragraph{Implement Revised Module System in Compiler Back-End} In this thesis we described and implemented a revised module system for WebDSL in Statix (see \cref{sec:module-system}) that can be described as a traditional module system where (with some exceptions) referencing a declaration made in another module requires importing that module. Additionally, it supports wildcard imports for pragmatic purposes during WebDSL development. Currently, the WebDSL compiler adheres to the module system as listed in the original WebDSL paper \autocite{Visser2007}, where it is described as follows: \textit{``a very simple module system has been chosen that supports distributing functionality over files, without separate compilation''}. 

    \paragraph{Connect Modernized Front-End to Existing Compiler Back-End} The WebDSL static analysis implemented in Stratego is not solely used for error reporting in the IDE. It generates signatures for definitions as discussed in \cref{sec:statix-string-manipulation} for which code should be generated, and it creates dynamic rules with name binding and type checking information on which the code generator depends. Re-implementing the current WebDSL back-end to use the Statix analysis results from the Statix Stratego API\footnote{https://www.spoofax.dev/references/statix/stratego-api/} is a possibility, but this may require a significant amount of time. An alternative is to write a connecting piece of software in Stratego that takes the Statix analysis result as input and generates the correct dynamic rules and AST terms such that the current WebDSL back-end can largely be used as is.

    In the current Stratego implementation, the code generation phase and type-checking phase are intertwined; additional WebDSL snippets are generated based on typing information, which could have non-local consequences and requires a type check of the program again \autocite{HemelKGV10}. This process is currently not implemented in the modernized static analysis, since it is not necessary for presenting the developer with early feedback. When the front-end is being connected to the back-end, analyzing the program multiple times with added generated code would be necessary. We suspect this would increase the run time of the whole compilation chain if the incremental Statix solver by Zwaan et al. (\citeyear{ZwaanAV22}) is utilized.

    \paragraph{Evaluate the Incremental Statix Solver} The recently published incremental Statix solver by Zwaan et al. (\citeyear{ZwaanAV22}) is promising in terms of speeding up the run time of executing Statix specifications. Their work already uses the WebDSL specification that we presented in this thesis, but it would be interesting to further evaluate the run time on other (closed-source) applications such as WebLab and conf.researchr.org, and the impact of specific WebDSL language constructs on the incrementality.

    \paragraph{Evaluate the Impact of Post-Parse Filters \texttt{\{prefer\}} and \texttt{\{avoid\}} on Parsing Run Time} In this thesis we have shown how we disambiguated the WebDSL grammar in SDF3 without the use of post-parse filters (see \cref{sec:webdsl-sdf3-disambiguation}). However, we do not provide insight into the impact of \texttt{\{prefer\}} and \texttt{\{avoid\}} on the run time of the generated parser, because we developed the SDF3 grammar from the start with the intention of leaving out the post-parse filters.
